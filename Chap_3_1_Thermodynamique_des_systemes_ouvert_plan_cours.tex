\documentclass[french]{article}
\usepackage[utf8]{inputenc}
\usepackage[T1]{fontenc}
\usepackage{lmodern}
\usepackage[a4paper]{geometry}
\usepackage{babel}
\usepackage{graphicx}

\usepackage{amsmath}
\usepackage{cancel}
\usepackage{tikz}
\usepackage{upgreek}
\usepackage{enumitem}

\begin{document}
	
\title{Chapitre 3.1. \\
Thermodynamique des systèmes ouverts en régime stationnaire}
\author{plan - Cours}
\date{}
	
\maketitle

\section*{Motivations}

\section{Formulation des principes de la thermodynamique pour une transformation élémentaire}

\subsection{Transformation élémentaire de grandeur thermodynamique}

\subsection{Premier principe d'une transformation élémentaire}

\subsection{Second principe pour une transformation élémentaire}

\section{Principes de la thermodynamique pour un système ouvert}

\subsection{Ecoulement stationnaire à travers un système ouvert}

\subsection{Bilan de masse, débit massique}

\subsection{Premier principe pour un système ouvert, Bilan d'énergie}

\subsection{Deuxième principe pour un système ouvert, Bilan d'entropie}

\section{Etude de machines thermique à l'aide du diagramme $(\ln(p), h)$ ou diagramme des frigoristes}

\subsection{Diagramme des frigoristes $(\ln(p),h)$ et machines élémentaires}



\end{document}